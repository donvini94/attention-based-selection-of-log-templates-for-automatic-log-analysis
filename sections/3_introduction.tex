%% LaTeX2e class for student theses
%% sections/content.tex
%% 
%% Karlsruhe Institute of Technology
%% Institute for Program Structures and Data Organization
%% Chair for Software Design and Quality (SDQ)
%%
%% Dr.-Ing. Erik Burger
%% burger@kit.edu
%%
%% Version 1.3.5, 2020-06-26

% 1. What's known 
% 2. What's unknown 
%     - limitations and gaps in previous studies
% 3. Your burning question/hypothesis/aim 
% 4. Your experimental approach 
% 5. Why your experimental approach is new and different and important (fill in the gaps)

\chapter{Introduction}
\label{ch:Introduction}
The increasing complexity of cloud systems and the vast amount of log data generated by these systems have led to a growing need for automated log analysis methods. Log parsing algorithms serve as a crucial component in this process, enabling the extraction of useful information from raw log data for various tasks, such as anomaly detection\cite{nedelkoski2020selfsupervised, bogatinovski2021multisource} and system monitoring \cite{10.1007/978-3-030-44769-4_13}. While numerous log parsing algorithms have been developed and extensively evaluated in the literature \cite{zhu2019tools}, selecting the most appropriate algorithm and its optimal parameterization for a specific task remains a challenging problem. Moreover, the same algorithm is often applied uniformly across the entire dataset, which may lead to sub optimal results. \\

Recently, an attention-based method for automating the selection of log parsing algorithms was proposed in a Master Thesis \cite{witterauf2021domainml}, demonstrating promising results with a single dataset and log parsing algorithm. However, it remains unknown whether this attention-based approach can generalize to other algorithms and datasets or if it can be extended to combine the results of multiple log parsers to further improve performance.\\

In this thesis, we aim to investigate the generalizability and applicability of the attention-based method across different log parsing algorithms and datasets. Our hypothesis is that this approach can be successfully employed as an industry-standard practice for automated log analysis. To test this hypothesis, we extend this approach with two additional log parsing algorithms and datasets.\\

Our experimental approach is novel and important for several reasons. First, it addresses the limitations of existing log parsing approaches, which often apply a single algorithm uniformly across the dataset, by utilizing an attention-based method to adaptively select the most suitable log parsing algorithm and parameterization for a given task. Second, our approach systematically explores the combination of multiple log parsers and their parameterizations, providing a more comprehensive evaluation of the attention-based method's potential to improve log analysis outcomes. Lastly, by examining the performance of this method across different algorithms and datasets, we contribute to the understanding of its generalizability and applicability in various log analysis scenarios, potentially paving the way for its adoption as an industry-standard practice.
