%% LaTeX2e class for student theses
%% sections/abstract_en.tex
%% 
%% Karlsruhe Institute of Technology
%% Institute for Program Structures and Data Organization
%% Chair for Software Design and Quality (SDQ)
%%
%% Dr.-Ing. Erik Burger
%% burger@kit.edu
%%
%% Version 1.3.5, 2020-06-26

\Abstract

Log analysis serves as a crucial preprocessing step in text log data analysis, including anomaly detection in cloud system monitoring. However, selecting an optimal log parsing algorithm tailored to a specific task remains problematic.

With many algorithms to choose from, each requiring proper parameterization, making an informed decision becomes difficult. Moreover, the selected algorithm is typically applied uniformly across the entire dataset, regardless of the specific data analysis task, often leading to suboptimal results. \\

In this thesis, we evaluate a novel attention-based method for automating the selection of log parsing algorithms, aiming to improve data analysis outcomes. We build on the success of a recent Master Thesis, which introduced this attention-based method and demonstrated its promising results for a specific log parsing algorithm and dataset. The primary objective of our work is to evaluate the effectiveness of this approach across different algorithms and datasets. \\

To accomplish this, we conduct extensive experiments to test the applicability and generalizability of the attention-based method in log analysis. By comparing multiple log parsing algorithms, we provide comprehensive evaluation within the context of various datasets. 
By addressing the limitations of existing log parsing approaches and proposing an attention-based method, we contribute to the advancement of automated log analysis. Furthermore, we provide practical guidelines on employing this method in log analysis projects, offering insights into how to approach log parsing algorithm selection for specific tasks.