%% LaTeX2e class for student theses
%% sections/abstract_de.tex
%% 
%% Karlsruhe Institute of Technology
%% Institute for Program Structures and Data Organization
%% Chair for Software Design and Quality (SDQ)
%%
%% Dr.-Ing. Erik Burger
%% burger@kit.edu
%%
%% Version 1.3.5, 2020-06-26

\Abstract

Die Log-Analyse dient als entscheidender Vorverarbeitungsschritt in der Text-Log-Datenanalyse, einschließlich der Anomalieerkennung in der Überwachung von Cloud-Systemen. Die Auswahl eines optimalen, auf eine spezifische Aufgabe zugeschnittenen Log-Parsing-Algorithmus bleibt jedoch problematisch.\\

Bei vielen zur Auswahl stehenden Algorithmen, die jeweils eine korrekte Parametrisierung erfordern, wird eine fundierte Entscheidung schwierig. Darüber hinaus wird der ausgewählte Algorithmus in der Regel einheitlich auf den gesamten Datensatz angewendet, unabhängig von der spezifischen Datenanalyseaufgabe, was oft zu suboptimalen Ergebnissen führt. \\

In dieser Arbeit bewerten wir eine neuartige, aufmerksamkeitsbasierte Methode zur Automatisierung der Auswahl von Log-Parsing-Algorithmen mit dem Ziel, die Ergebnisse der Datenanalyse zu verbessern. Wir bauen auf dem Erfolg einer kürzlichen Masterarbeit auf, die diese aufmerksamkeitsbasierte Methode eingeführt und ihre vielversprechenden Ergebnisse für einen spezifischen Log-Parsing-Algorithmus und Datensatz demonstriert hat. Das Hauptziel unserer Arbeit besteht darin, die Wirksamkeit dieses Ansatzes bei verschiedenen Algorithmen und Datensätzen zu bewerten. \\

Um dies zu erreichen, führen wir umfangreiche Experimente durch, um die Anwendbarkeit und Allgemeingültigkeit der aufmerksamkeitsbasierten Methode in der Log-Analyse zu testen. Durch den Vergleich mehrerer Log-Parsing-Algorithmen bieten wir eine umfassende Bewertung im Kontext verschiedener Datensätze. 
Indem wir die Einschränkungen bestehender Log-Parsing-Ansätze ansprechen und eine aufmerksamkeitsbasierte Methode vorschlagen, tragen wir zur Weiterentwicklung der automatisierten Log-Analyse bei. Darüber hinaus bieten wir praktische Leitlinien für den Einsatz dieser Methode in Log-Analyse-Projekten und geben Einblicke, wie die Auswahl von Log-Parsing-Algorithmen für spezifische Aufgaben angegangen werden kann.